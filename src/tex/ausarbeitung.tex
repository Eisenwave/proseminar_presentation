\documentclass{article}
\usepackage[utf8]{inputenc}
\usepackage{fancyhdr}
\usepackage[hidelinks]{hyperref}
\usepackage[ngerman]{babel}

%\usepackage{hyperref}
%\usepackage{titling}

%\newcommand{\subtitle}[1]{
%\posttitle{
%\par\end{center}
%\begin{center}\LARGE#1\end{center}
%\vskip0.5em}
%}

\title{\textbf{\LARGE Pseudozufallsfolgengeneratoren
\\{\Large Technische Universit\"at Dresden}
\\\noindent\newline{\large Proseminar Sicherheit in Computersystemen}
\\{\small Professur f\"ur Datenschutz und Datensicherheit}
}}
%\subtitle{Pseudozuffalsfolgengeneratoren}

%\title{\LARGE Pseudozuffalsfolgengeneratoren}
\author{\\ \Large Jan Schultke}
\date{\large2019-06-06}

\pagestyle{fancy}
\fancyhf{}
\rhead{\nouppercase{\leftmark}}
\lhead{Pseudozufallsfolgengeneratoren}
%\rfoot{Page \thepage}
\cfoot{\thepage}

\setcounter{tocdepth}{3}
%\renewcommand{\footrulewidth}{0.5pt}
\renewcommand\refname{8 Literaturverzeichnis}
\renewcommand{\contentsname}{Inhaltsverzeichnis}
%\parskip = \baselineskip

\let\oldquote'
\newif\ifquoteopen
\catcode`\'=\active
\makeatletter
\DeclareRobustCommand*{'}{%
\@ifnextchar'{%
\ifquoteopen
\global\quoteopenfalse\grqq\expandafter\@gobble
\else
\global\quoteopentrue\glqq\expandafter\@gobble
\fi
}{\oldquote}%
}
\makeatother

\begin{document}

    %\section{Title}\label{sec:title}
    \maketitle
    \pagenumbering{gobble}
    \clearpage
    \pagebreak


    \section*{Abstract}\label{sec:abstract}
    %\begin{abstract}
    Pseudozufallsfolgengeneratoren (PZFG) sind eine seit den 1940ern weit verbreitete Methode zur Generierung von
    Zufallsfolgen in Computersystemen.
    Nahezu jede Programmiersprache und auch einige Betriebssysteme (u.A. Unix) machen Gebrauch von dieser Technologie.

    \noindent\newline Diese Arbeit vermittelt Grundlagen des Entwurfs von PZFG und gew\"unchste Eigenschaften und
    Funktionsweisen.
    Visuelle sowie stochastische Bewertungskriterien von PZFG werden am Beispiel von John von Neumanns
    Mittquadratmethode und dem seit 1960 eingesetzten RANDU-Algorithmus vorgestellt.
    Dies beinhalten unter anderem die Analyse von Werteverteilungen, Analyse der niederwertigsten Bitebene, Darstellung
    der Generatorfunktion in zwei und drei Dimensionen und die Erweiterung dieses Konzepts auf den Spektraltest.

    \noindent\newline Ebenso beinhaltet diese Arbeit einen Ausblick in moderne Verfahren und kryptographisch sichere PZFG.
    Prim\"ar wird hier die Tauglichkeit von Hashfunktionen sowie von Blockchiffren als PZFG betrachtet.
    %\end{abstract}
    \clearpage
    \pagebreak


    \pagenumbering{arabic}
    \tableofcontents
    \pagebreak



    %\addcontentsline{toc}{subsection}{\numberline{\thesection.2} Some Subsection}

    \section{Einf\"uhrung}\label{sec:intro}
    Zufall.
    In der Natur mit Leichtigkeit zu finden, wenn man nur dem Rauschen des Meeres lauscht oder die Verteilung von
    K\"ornern im Sand betrachtet.
    Mit einem W\"urfel oder Gl\"ucksrad kann er sogar zur beliebigen Zeit und in beliebigen Mengen erzeugt werden.
    Doch wie kann ein Computer eben diesen Zufall erzeugen, wenn er eine volkommen deterministische Maschine ist?

    \noindent\newline Ein erster Ansatz w\"are es diesen Zufall von Au{\ss}en in den Rechner einzuspeisen.
    1940 konstruierte die RAND-Corporation eine Maschine, welche mithilfe eines Pulsgenerators Zufallswerte
    erzeugte und ver\"offentlichte die Ergebnisse in \textit{''A Million Random Digits with 100,000 Normal Deviates''}\cite{random_digits}.
    Doch ob es sich bei der Quelle um Lochkarten eines Großrechners aus den 1960ern oder um DDR4-Speicher auf einer
    hochmodernen Hauptplatine handelt:
    Das Einspeisen von Zufall in den Rechner ist weit zeitintensiver als arithmetische Operationen und die verf\"ugbare
    Entropie ist immer endlich.

    \noindent\newline Alternativ kann jeder Prozessor mit einer Instruktion zum Generieren echten Zufalls ausgestattet
    werden.
    So besitzt der 1951er Ferranti Mark 1 eine von Alan Turing entwickelte Hardwareinstruktion\cite{ferranti_rng},
    welche mithilfe elektrischen Rauschens 20 echte Zufallsbits erzeugt.
    Jedoch gehen durch solche Instruktionen viele Qualit\"aten eines deterministischen Algorithmus verloren, wie zum
    Beispiel die Reproduzierbarkeit einer Zufallsfolge.

    \noindent\newline Abhilfe f\"ur die Probleme der vorigen Ans\"atze schaffen Pseudozufallsfolgengene-ratoren (PZFG).
    Dabei werden Folgen von zufallsverteilten Werten mithilfe von deter-ministischen Algorithmen erzeugt.
    Ob sie nun eingesetzt werden um einen Klartext mit Zufallsbits zu verschl\"usseln, eine Lanschaft prozedurell
    zu generieren, KI in Spielen unvorhersehbar handeln zu lassen oder Funktionen eines Unit-Tests mit
    (pseudo)zuf\"alligen Werten zu speisen, Zufallsgeneratoren sind in nahezu jedem Computersystem zu finden.

    \noindent\newline Ein erster Ansatz hierf\"ur ist die 1949 erstmals von John von Neumann vorgestellte
    Mittquadratmethode\cite{middle_square}.
    Im Kapitel 4 wird diese ausgiebig betrachtet und diverse Bewertungskriterien f\"ur PZFG werden vorgestellt.
    Von der Effizenz eines Algo-rithmus bis zu diversen stochastischen Tests werden Bewertungskriterien sowohl formal
    als auch visuell pr\"asentiert.
    In Kapitel 5 werden Kongruenzgeneratoren betrachtet;
    ein historisch sehr weit verbreiteter Ansatz zur Pseudozufallsgenerierung.
    In Kapitel 6 ein Ausblick in moderne Verfahren zu finden, insbesondere der Einsatz von
    Hashfunktionen und Blockchiffren als PZFG.
    \pagebreak

    \addcontentsline{toc}{section}{\numberline{3}Definitionen}

    \addcontentsline{toc}{section}{\numberline{4}John von Neumanns Mittquadratmethode:
    \\\small Die Geburt des Pseudozufalls}
    \addcontentsline{toc}{subsection}{\numberline{4.1}Geschichte und Anwendung}
    \addcontentsline{toc}{subsection}{\numberline{4.2}Funktionsweise und offensichtliche Schwachstellen}
    \addcontentsline{toc}{subsection}{\numberline{4.3}Bewertungskriterien von PZFG anhand der Mittquadratmethode}
    \addcontentsline{toc}{subsubsection}{\numberline{4.3.1}Effizienz}
    \addcontentsline{toc}{subsubsection}{\numberline{4.3.2}Werteverteilung: Wunsch und Realit\"at}
    \addcontentsline{toc}{subsubsection}{\numberline{4.3.3}Periodenl\"ange}
    \addcontentsline{toc}{subsubsection}{\numberline{4.3.4}LSB-Analyse: Muster in niederwertigsten Bits}
    \addcontentsline{toc}{subsubsection}{\numberline{4.3.5}Zeichnen der Funktion: Ein visueller Test}
    \addcontentsline{toc}{subsubsection}{\numberline{4.3.6}Ausblick in diverse weitere Stochastische Tests}

    \addcontentsline{toc}{section}{\numberline{5}Kongruenzgeneratoren:
    \\\small Pseudozufall in der Praxis}
    \addcontentsline{toc}{subsection}{\numberline{5.1}Definitionen: Vom Multiplikativen zum Allgemeinen}
    \addcontentsline{toc}{subsection}{\numberline{5.2}RANDU: primitivste Pseudozufallsgenerierung}
    \addcontentsline{toc}{subsection}{\numberline{5.3}Bewertungskriterien von PZFG anhand von RANDU}
    \addcontentsline{toc}{subsubsection}{\numberline{5.3.1}Effizienz: RANDU mit nur Addition und Bitoperationen}
    \addcontentsline{toc}{subsubsection}{\numberline{5.3.2}Werteverteilung und Periodenl\"ange: RANDU und
    Kongruenz-generatoren im Allgemeinen}
    \addcontentsline{toc}{subsubsection}{\numberline{5.3.3}LSB-Analyse: Muster in niederwertigsten Bits}
    \addcontentsline{toc}{subsubsection}{\numberline{5.3.4}Spektraltest: RANDUs 15 Ebenen}

    \addcontentsline{toc}{section}{\numberline{6}Moderne Verfahren:
    \\\small Pseudozufall mit Qualit\"at an erster Stelle}
    \addcontentsline{toc}{subsection}{\numberline{6.1}Zuverl\"assige PZFGeneratoren}
    \addcontentsline{toc}{subsubsection}{\numberline{6.1.1}Hashfunktionen als PZFGeneratoren}
    \addcontentsline{toc}{subsection}{\numberline{6.2}Kryptographisch sichere PZFGeneratoren}
    \addcontentsline{toc}{subsubsection}{\numberline{6.2.1}Blockchiffren am Beispiel des AES}

    \addcontentsline{toc}{section}{\numberline{7} Schlussfolgerung}
    \addcontentsline{toc}{section}{\numberline{8} Literaturverzeichnis}

    %\paragraph{Outline}
    %\section*{8 Literaturverzeichnis}\label{sec:bibliography}
    %Section~\ref{sec:previous work} gives account of previous work.
    %Our new and exciting results are described in Section~\ref{sec:results}.
    %Finally, Section~\ref{sec:conclusions} gives the conclusions.
    %    \section{Previous work}\label{sec:previous work}
    %    A much longer \LaTeXe{} example was written by Gil~\cite{Gil:02}.

    \bibliographystyle{abbrv}
    \begin{thebibliography}{9}
        \bibitem{random_digits}
        RAND:
        \textit{A Million Random Digits with 100,000 Normal Deviates} (1955)
        \\\texttt{\url{https://www.rand.org/content/dam/rand/pubs/monograph_reports/MR1418/MR1418.deviates.pdf}}
        (abgerufen 2019-06-01)

        \bibitem{ferranti_rng}
        A. M. Turing, R. A. Brooker:
        \textit{Programmer's Handbook for the Manchester Electronic Computer Mark II (2. Auflage)} 1. (1952)
        \\\texttt{\url{http://curation.cs.manchester.ac.uk/computer50/www.computer50.org/kgill/mark1/progman.html}}
        (abgerufen 2019-06-01)

        \bibitem{middle_square}
        Die 1949er Papiere wurden erst 1951 neu gedruckt:
        \\J. Neumann, Von:
        \textit{Various Techniques Used in Connection with Random Digits}
        (1951, National Bureau of Standards, Applied Math Series)
        \\\texttt{\url{https://mcnp.lanl.gov/pdf_files/nbs_vonneumann.pdf}}
        (abgerufen 2019-06-01)

        \bibitem{latexcompanion}
        W. Killmann, W. Schindler; Bundesamt f\"ur Sicherheit in der Informationstechnik:
        \textit{A proposal for Functionality classes for random number generators} (2011-09-18)
        \\\texttt{\url{https://www.bsi.bund.de/SharedDocs/Downloads/DE/BSI/Zertifizierung/Interpretationen/AIS_31_Functionality_classes_for_random_number_generators_e.pdf?__blob=publicationFile}}
        (abgerufen 2019-04-30)
        %Addison-Wesley, Reading, Massachusetts, 1993.

        \bibitem{seminumerical algorithms}
        D. E. Knuth:
        \textit{The Art of Computer Programming, Volume 2: Seminumerical Algorithms} (1997, Addison-Wesley)
        \\\textttt{\url{https://doc.lagout.org/science/0_Computer%20Science/2_Algorithms/The%20Art%20of%20Computer%20Programming%20%28vol.%202_%20Seminumerical%20Algorithms%29%20%283rd%20ed.%29%20%5BKnuth%201997-11-14%5D.pdf
        }}
        (abgerufen 2019-06-01)

        \bibitem{pokemon}
        ''Fractal Fusion'':
        \textit{Pokemon RBGY Random Number Generator}
        \\\texttt{\url{https://web.archive.org/web/20190105083840/http://tasvideos.org/GameResources/GBx/Pokemon.html}}
        (abgerufen 2019-06-01)

        \bibitem{fortran}
        Compaq:
        \textit{Compaq Fortran Language Reference Manual}
        \\\texttt{\url{http://h30266.www3.hpe.com/odl/unix/progtool/cf95au56/lrm0315.htm#randu_intrin}}
        (abgerufen 2019-05-04)

        \bibitem{neumann}
        J. Neumann, Von:
        \textit{Various techniques used in connection with random digits}, 12. 36-38
        (1951, National Bureau of Standards, Applied Mathematics Series)
        \\\texttt{\url{https://dornsifecms.usc.edu/assets/sites/520/docs/VonNeumann-ams12p36-38.pdf}}
        (abgerufen 2019-04-30)

        \bibitem{einstein}
        Frank Yellin et al. f\"ur Oracle:
        \textit{Random.java} (2014-03-04, OpenJDK)
        \\\texttt{\url{http://hg.openjdk.java.net/jdk8/jdk8/jdk/file/tip/src/share/classes/java/util/Random.java}}
        (abgerufen 2019-04-30)

        \bibitem{aes}
        N. Ferguson, B. Schneier, T. Kohno:
        \textit{Cryptography Engineering: Design Principles and Practical Applications} (2010, Wiley Publishing)
        \\\texttt{\url{https://www.schneier.com/academic/paperfiles/fortuna.pdf}}
        (abgerufen 2019-06-01)

    \end{thebibliography}

\end{document}
