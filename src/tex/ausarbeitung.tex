\documentclass{article}
\usepackage[utf8]{inputenc}
\usepackage{fancyhdr}
\usepackage[hidelinks]{hyperref}

%\usepackage{hyperref}
%\usepackage{titling}

%\newcommand{\subtitle}[1]{
%\posttitle{
%\par\end{center}
%\begin{center}\LARGE#1\end{center}
%\vskip0.5em}
%}

\title{\textbf{\LARGE Pseudozufallsfolgengeneratoren
\\{\large Technische Universit\"at Dresden}
}}
%\subtitle{Pseudozuffalsfolgengeneratoren}

%\title{\LARGE Pseudozuffalsfolgengeneratoren}
\author{\Large Jan Schultke}
\date{\large2019-04-03}

\pagestyle{fancy}
\fancyhf{}
\rhead{\nouppercase{\leftmark}}
\lhead{Pseudozuffalsfolgengeneratoren}
%\rfoot{Page \thepage}
\cfoot{\thepage}

\setcounter{tocdepth}{3}
%\renewcommand{\footrulewidth}{0.5pt}
\renewcommand\refname{8 Literaturverzeichnis}

%\parskip = \baselineskip

\begin{document}

    %\section{Title}\label{sec:title}
    \maketitle
    \pagenumbering{gobble}
    \clearpage
    \pagebreak

    \pagenumbering{arabic}
    \tableofcontents
    \pagebreak

    \section{Abstract}\label{sec:abstract}
    %\begin{abstract}
    Pseudozufallsfolgengeneratoren (PZFG) sind eine seit den 1940ern weit verbreitete Methode zur Generierung von
    Zufallsfolgen in Computersystemen.
    Nahezu jede Programmiersprache und auch einige Betriebssysteme (u.A. Unix) machen Gebrauch von dieser Technologie.

    \noindent\newline Diese einf\"uhrende Arbeit vermittelt Grundlagen des Entwurfs von PZFGs und gew\"unchste Eigenschaften und
    Funktionsweisen.
    Visuelle sowie stochastische Bewertungskriterien von PZFGs werden am Beispiel von John von Neumanns
    Mittelquadratmethode und dem seit 1960 eingesetzten RANDU-Algorithmus vorgestellt.
    Dies beinhalten unter anderem die Analyse von Werteverteilungen, Analyse der niederwertigsten Bitebene, Darstellung
    der Generatorfunktion in zwei und drei Dimensionen und die Erweiterung dieses Konzepts auf den Spektraltest.

    \noindent\newline Ebenso beinhaltet diese Arbeit einen Ausblick in moderne Verfahren und kryptographisch sichere PZFGs.
    Prim\"ar wird hier die Tauglichkeit von Hashfunktionen sowie von Blockchiffren als PZFG betrachtet.
    %\end{abstract}
    \pagebreak

    %\addcontentsline{toc}{subsection}{\numberline{\thesection.2} Some Subsection}

    \section{Einf\"uhrung}\label{sec:intro}
    Zufall.
    In der Natur mit Leichtigkeit zu finden, wenn man nur dem Rauschen des Meeres lauscht oder die Verteilung von
    K\"ornern im Sand betrachtet.
    Mit einem W\"urfel oder Gl\"ucksrad kann er sogar zur beliebigen Zeit und in beliebigen Mengen erzeugt werden.
    Doch wie kann ein Computer eben diesen Zufall erzeugen, wenn er eine volkommen deterministische Maschine ist?

    \noindent\newline Ein erster Ansatz w\"are es diesen Zufall von Au{\ss}en in den Rechner einzuspeisen.
    1940 entwickelte die RAND-Corporation eine spezielle Maschine die mithilfe eines Pulsgenerators Zufallswerte
    erzeugt und ver\"offentlichte diese in \textit{"A Million Random Digits with 100,000 Normal Deviates"}.
    Doch ob es sich um Lochkarten eines Prozessors aus den 1960ern oder um DDR4-Speicher eines 2019er Prozessors
    handelt, das Einspeisen dieser Werte dauert immer noch l\"anger als einfache mathematische Operationen und
    irgendwann gehen diese Werte auch zu neige.

    \noindent\newline Auch kann jeder Rechner mit einer Instruktion zum Generieren echten Zufalls ausgestattet werden.
    So besitzt der 1951er Ferranti Mark 1 eine von Alan Turing entwickelte Hardwareinstruktion, welche mithilfe
    elektrischen Rauschens echte 20 Zufallsbits erzeugt.
    Jedoch gehen durch solche Instruktionen viele klassiche Qualit\"aten eines deterministischen Algorithmus verloren
    und die Implementierung erfordert zus\"atzliche Peripherieger\"ate.

    \noindent\newline Eine L\"osung hierf\"ur sind Pseudozufallsfolgengeneratoren (PZFG).
    Dabei werden scheinbar zuf\"allige Werte mithilfe von deterministischen Algorithmen erzeugt.
    Ob sie nun eingesetzt werden um einen Klartext mit Zufallsbits zu verschl\"usseln, eine Lanschaft prozedurell
    zu generieren, KI in Spielen unvorhersehbar handeln zu lassen oder Funktionen eines Unit-Tests mit zuf\"allig
    erzeugten Werten zu speisen, Zufallsgeneratoren sind in so ziemlich jedem Computersystem zu finden.

    \noindent\newline Ein erster Ansatz hierf\"ur ist die 1949 erstmals von John von Neumann vorgestellte
    Mittelquadratmethode.
    Im Kapitel 4 wird diese ausgiebig betrachtet und diverse Bewertungskriterien f\"ur PZFGs werden vorgestellt.
    Von der Effizenz eines Algorithmus bis zu diversen stochastischen Tests werden Bewertungskriterien sowohl formal
    als auch visuell pr\"asentiert.
    In Kapitel 5 werden weiterhin Kongruenzgeneratoren betrachtet, welche bei weitem in historischen als auch modernen
    Computersystemen am Meisten implementiert wurden.
    Zu guter Letzt ist in Kapitel 6 ein Ausblick in moderne Verfahren zu finden, insbesondere der Einsatz von
    Hashfunktionen und Blockchiffren.
    \pagebreak

    \addcontentsline{toc}{section}{\numberline{3}Definitionen}

    \addcontentsline{toc}{section}{\numberline{4}John von Neumanns Mittelquadratmethode:
    \\\small Die Geburt des Pseudozufalls}
    \addcontentsline{toc}{subsection}{\numberline{4.1}Geschichte und Anwendung}
    \addcontentsline{toc}{subsection}{\numberline{4.2}Funktionsweise und offensichtliche Schwachstellen}
    \addcontentsline{toc}{subsection}{\numberline{4.3}Bewertungskriterien von PZFG anhand der Mittelquadratmethode}
    \addcontentsline{toc}{subsubsection}{\numberline{4.3.1}Effizienz}
    \addcontentsline{toc}{subsubsection}{\numberline{4.3.2}Werteverteilung: Wunsch und Realit\"at}
    \addcontentsline{toc}{subsubsection}{\numberline{4.3.3}Periodenl\"ange}
    \addcontentsline{toc}{subsubsection}{\numberline{4.3.4}LSB-Analyse: Muster in niederwertigsten Bits}
    \addcontentsline{toc}{subsubsection}{\numberline{4.3.5}Zeichnen der Funktion: Ein visueller Test}
    \addcontentsline{toc}{subsubsection}{\numberline{4.3.6}Ausblick in diverse weitere Stochastische Tests}

    \addcontentsline{toc}{section}{\numberline{5}Kongruenzgeneratoren:
    \\\small Pseudozufall in der Praxis}
    \addcontentsline{toc}{subsection}{\numberline{5.1}Definitionen: Vom Multiplikativen zum Allgemeinen}
    \addcontentsline{toc}{subsection}{\numberline{5.2}RANDU: primitivste Pseudozufallsgenerierung}
    \addcontentsline{toc}{subsection}{\numberline{5.3}Bewertungskriterien von PZFG anhand von RANDU}
    \addcontentsline{toc}{subsubsection}{\numberline{5.3.1}Effizienz: RANDU mit nur Addition und Bitoperationen}
    \addcontentsline{toc}{subsubsection}{\numberline{5.3.2}Werteverteilung und Periodenl\"ange: RANDU und
    Kongruenzgeneratoren im Allgemeinen}
    \addcontentsline{toc}{subsubsection}{\numberline{5.3.3}LSB-Analyse: Muster in niederwertigsten Bits}
    \addcontentsline{toc}{subsubsection}{\numberline{5.3.4}Spektraltests: RANDUs 15 Ebenen}

    \addcontentsline{toc}{section}{\numberline{6}Moderne Verfahren:
    \\\small Pseudozufall mit Qualit\"at an erster Stelle}
    \addcontentsline{toc}{subsection}{\numberline{6.1}Zuverl\"assige PZFGeneratoren}
    \addcontentsline{toc}{subsubsection}{\numberline{6.1.1}Hashfunktionen als PZFGeneratoren}
    \addcontentsline{toc}{subsection}{\numberline{6.2}Kryptographisch sichere PZFGeneratoren}
    \addcontentsline{toc}{subsubsection}{\numberline{6.2.1}Blockchiffren am Beispiel des AES}

    \addcontentsline{toc}{section}{\numberline{7} Schlussfolgerung}
    \addcontentsline{toc}{section}{\numberline{8} Literaturverzeichnis}

    %\paragraph{Outline}
    %\section*{8 Literaturverzeichnis}\label{sec:bibliography}
    %Section~\ref{sec:previous work} gives account of previous work.
    %Our new and exciting results are described in Section~\ref{sec:results}.
    %Finally, Section~\ref{sec:conclusions} gives the conclusions.
    %    \section{Previous work}\label{sec:previous work}
    %    A much longer \LaTeXe{} example was written by Gil~\cite{Gil:02}.

    \bibliographystyle{abbrv}
    \begin{thebibliography}{9}
        \bibitem{latexcompanion}
        W. Killmann, W. Schindler; Bundesamt f\"ur Sicherheit in der Informationstechnik:
        \textit{A proposal for Functionality classes for random number generators} (2011-09-18)
        \\\texttt{\url{https://www.bsi.bund.de/SharedDocs/Downloads/DE/BSI/Zertifizierung/Interpretationen/AIS_31_Functionality_classes_for_random_number_generators_e.pdf;jsessionid=90B3596E4BF995CBF33B698F0C6723D9.1_cid360?__blob=publicationFile&v=1}}
        (abgerufen 2019-04-30)
        %Addison-Wesley, Reading, Massachusetts, 1993.

        \bibitem{seminumerical algorithms}
        D. E. Knuth:
        \textit{The Art of Computer Programming, Volume 2: Seminumerical Algorithms} (1997)
        \\ Addison-Wesley

        \bibitem{neumann}
        J. Von Neumann:
        \textit{Various techniques used in connection with random digits} (1951)
        \\\texttt{\url{https://dornsifecms.usc.edu/assets/sites/520/docs/VonNeumann-ams12p36-38.pdf}}
        (abgerufen 2019-04-30)
        \\ National Bureau of Standards Applied Mathematics Series

        \bibitem{einstein}
        katleman; OpenJDK:
        \textit{Random.java} (2014-03-04)
        \\\texttt{\url{http://hg.openjdk.java.net/jdk8/jdk8/jdk/file/tip/src/share/classes/java/util/Random.java}}
        (abgerufen 2019-04-30)
        \\ OpenJDK

        \bibitem{aes}
        N. Ferguson, B. Schneier, T. Kohno:
        \textit{Cryptography Engineering: Design Principles and Practical Applications} (2010)
        \\\texttt{\url{https://www.schneier.com/academic/paperfiles/fortuna.pdf}}
        \\ Wiley Publishing
    \end{thebibliography}

\end{document}
