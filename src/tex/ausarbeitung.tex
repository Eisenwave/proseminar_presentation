\documentclass{article}
\usepackage[utf8]{inputenc}
\usepackage{fancyhdr}

%\usepackage{hyperref}
%\usepackage{titling}

%\newcommand{\subtitle}[1]{
%\posttitle{
%\par\end{center}
%\begin{center}\LARGE#1\end{center}
%\vskip0.5em}
%}

\title{\textbf{\LARGE Pseudozufallsfolgengeneratoren
\\{\large Technische Universit\"at Dresden}
}}
%\subtitle{Pseudozuffalsfolgengeneratoren}

%\title{\LARGE Pseudozuffalsfolgengeneratoren}
\author{\Large Jan Schultke}
\date{\large\today}

\pagestyle{fancy}
\fancyhf{}
\rhead{\nouppercase{\leftmark}}
\lhead{Pseudozuffalsfolgengeneratoren}
%\rfoot{Page \thepage}
\cfoot{\thepage}

\setcounter{tocdepth}{3}
%\renewcommand{\footrulewidth}{0.5pt}

\begin{document}

    %\section{Title}\label{sec:title}
    \maketitle
    \pagenumbering{gobble}
    \clearpage
    \pagebreak

    \pagenumbering{arabic}
    \tableofcontents
    \pagebreak

    \section{Abstract}\label{sec:abstract}
    %\begin{abstract}
    This is the paper's abstract \ldots
    %\end{abstract}
    \pagebreak

    %\addcontentsline{toc}{subsection}{\numberline{\thesection.2} Some Subsection}

    \section{Einf\"uhrung}\label{sec:intro}
    This is time for all good men to come to the aid of their party!
    \pagebreak

    \addcontentsline{toc}{section}{\numberline{3}Definitionen}

    \addcontentsline{toc}{section}{\numberline{4}John von Neumanns Mittelquadratmethode:
    \\\small Die Geburt des Pseudozufalls}
    \addcontentsline{toc}{subsection}{\numberline{4.1}Geschichte und Anwendung}
    \addcontentsline{toc}{subsection}{\numberline{4.2}Funktionsweise und offensichtliche Schwachstellen}
    \addcontentsline{toc}{subsection}{\numberline{4.3}Bewertungskriterien von PZFG anhand der Mittelquadratmethode}
    \addcontentsline{toc}{subsubsection}{\numberline{4.3.1}Effizienz}
    \addcontentsline{toc}{subsubsection}{\numberline{4.3.2}Werteverteilung: Wunsch und Realit\"at}
    \addcontentsline{toc}{subsubsection}{\numberline{4.3.3}Periodenl\"ange}
    \addcontentsline{toc}{subsubsection}{\numberline{4.3.4}LSB-Analyse: Muster in niederwertigsten Bits}
    \addcontentsline{toc}{subsubsection}{\numberline{4.3.5}Zeichnen der Funktion: Ein visueller Test}
    \addcontentsline{toc}{subsubsection}{\numberline{4.3.6}Ausblick in diverse weitere Stochastische Tests}

    \addcontentsline{toc}{section}{\numberline{5}Kongruenzgeneratoren:
    \\\small Pseudozufall in der Praxis}
    \addcontentsline{toc}{subsection}{\numberline{5.1}Definitionen: Vom Multiplikativen zum Allgemeinen}
    \addcontentsline{toc}{subsection}{\numberline{5.2}RANDU: primitivste Pseudozufallsgenerierung}
    \addcontentsline{toc}{subsection}{\numberline{5.3}Bewertungskriterien von PZFG anhand von RANDU}
    \addcontentsline{toc}{subsubsection}{\numberline{5.3.1}Effizienz: RANDU mit nur Addition und Bitoperationen}
    \addcontentsline{toc}{subsubsection}{\numberline{5.3.2}Werteverteilung und Periodenl\"ange: RANDU und
    Kongruenzgeneratoren im Allgemeinen}
    \addcontentsline{toc}{subsubsection}{\numberline{5.3.3}LSB-Analyse: Muster in niederwertigsten Bits}
    \addcontentsline{toc}{subsubsection}{\numberline{5.3.4}Spektraltests: RANDUs 15 Ebenen}

    \addcontentsline{toc}{section}{\numberline{6}Moderne Verfahren:
    \\\small Pseudozufall mit Qualit\"at an erster Stelle}
    \addcontentsline{toc}{subsection}{\numberline{6.1}Zuverl\"assige PZFGeneratoren}
    \addcontentsline{toc}{subsubsection}{\numberline{6.1.1}Hashfunktionen als PZFGeneratoren}
    \addcontentsline{toc}{subsection}{\numberline{6.2}Kryptographisch sichere PZFGeneratoren}
    \addcontentsline{toc}{subsubsection}{\numberline{6.2.1}Blockchiffren am Beispiel des AES}

    \addcontentsline{toc}{section}{\numberline{7} Schlussfolgerung}
    \addcontentsline{toc}{section}{\numberline{8} Literaturverzeichnis}

    %\paragraph{Outline}
    .
    %Section~\ref{sec:previous work} gives account of previous work.
    %Our new and exciting results are described in Section~\ref{sec:results}.
    %Finally, Section~\ref{sec:conclusions} gives the conclusions.
    %    \section{Previous work}\label{sec:previous work}
    %    A much longer \LaTeXe{} example was written by Gil~\cite{Gil:02}.

    \bibliographystyle{abbrv}
    \bibliography{main}

\end{document}
